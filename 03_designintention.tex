\section*{Utopier}
Dorrestijn og Verbeek præsenterer et syn på, hvordan en higen efter det utopiske gennem design kan gøres relevant i vores samtid~\cite[]{dorrestijn2013technology} . Her præsenteres \textit{Nudging} og \textit{Persuasive Technology} som de to primære tilgange til adfærdsændring af folk gennem design. Dorrestijn og Verbeek foreslår en designmetodik, der fordrer \textit{opting-in} snarere end \textit{opting-out}, altså et tilvalg snarere end et fravalg. Denne måde at tænke på kan oversættes ganske nøjagtigt til problemstillingen om brugen af \textit{Big Data} til overvågning.

Nutidige ordensmagters tilgang til forhindring af kriminalitet er meget lig tilgangen i Nudging og Persuasive Technology. Brugen af vores data til det formål at beskytte os har skabt løsninger, der overser det enkelte menneskes behov og i stedet ser på os som dataindgange i et system, der så effektivt som muligt skal bekæmpe kriminalitet og skabe sikkerhed og tryghed. Dette systems syn på kriminelle er netop synet på kriminalitet som et tilvalg og altså synet på det retfærdige liv som fravalget af kriminalitet. Grundet dette syn bliver kriminalitet en handling, der skal forhindres og en kriminel, der jo selv har valgt kriminaliteten til, skal straffes eller i hvert fald rehabiliteres. Dette ses især tydeligt i PredPol-systemet, der netop hævder at kunne komme kriminalitet til livs ved at vide, hvor og hvornår den vil finde sted og derved forhindre den. På samme måde bliver tryghed en følelse som systemet udbyder. Denne følelse kan fravælges ved at stille sig uden for systemet (og blive kriminel), men den kan aldrig tilvælges. Som modpol, vil vi opstille et utopisk scenarie, der netop giver den enkelte borger mulighed for aktivt at deltage i et fællesskab, der i samarbejde opbygger en kultur omkring brugen af Big Data, og derved mulighed for at tilvælge sin egen retfærdighed og tryghed.

\section*{[a]- \& [b]-listerne}
For at nærmere at belyse den praktiske og politiske situation omkring brug af Big Data til at skabe tryghed, har vi fundet inspiration i Dunne \& Raby's [a]- \& [b]-lister \cite[]{ABList}. Først vil vi præsentere en [a]-liste, der eksemplificerer den praksis vores samfund ultimativt bevæger sig imod ved at  opsummere kendetegnene ved den tilgang, systemet i dag har til Big Data som værktøj til at skabe tryghed. Senere vil vi præsentere en [b]-liste, der opsummerer en modpol til [a]-listen og tegner en anderledes utopi for brugen af Big Data.

\subsection*{[a]-listen}
PredPol, CityPulse og snart Pol-Intel bygger alle på samme præmis. Ved at være tilstede, hvor og når kriminaliteten sker, kan man afværge forbrydelser og uro. Ved at afværge forbrydelser, vil kriminaliteten falde. Myndighederne skal altså kunne træde ind og \textit{beskytte} borgerne mod kriminelle handlinger, der vil give dem ubehagelige og utrygge oplevelser. %Følges denne tilgang til dør, bliver det myndighedernes opgave at beskytte befolkningen, der ultimativt aldrig vil opleve kriminaliteten, fordi myndighederne kan stoppe den, før den er sket. 
Ultimativt giver det myndighederne et monopol på tryghed, da borgeren aldrig får muligheden for at skabe deres egen tryghed. Alt indsamlet data ejes af systemet og bruges af den lovopretholdende indstands, der forudsiger kriminaliteten ved at analysere de store mængder data for at pege på hvem, hvornår, hvor og hvad. %Dette fører til en stigmatisering, hvor myndigheder bestemmer, hvem der er kriminelle, da de har eneret på denne data. 
Borgerne kan blot forsøge at holde sig i mængden af de lovlydige, så de ikke kommer i myndighedernes søgelys og potentielt bliver udpeget som kriminelle. Nedenfor har vi listet disse pointer i en [a]-liste.

\begin{figure}[H]
        \begin{itemize}
            \item[] \textbf{[a]}
            \item Tryghed oppefra
            \item Tryghed er tilliden til systemet
            \item Symptombehandling
            \item Skabe ro og orden
            \item Systemet beskytter individet
            \item Systemet afværger kriminalitet
            \item Systemet profilerer individer
            \item Data ejes af systemet
            \item Data bruges af systemet
            \item Kortsigtet~\textrightarrow~nu og her
        \end{itemize}

    \caption{[a]
    -listen}
    \label{fig:a_liste}
\end{figure}


For at få en dybere forståelse af denne praksis, kan vi bruge Meadows' Leverage Points~\cite[]{LeveragePoints}. Der ses en tendens til generelt at bruge Big Data til løsninger med et kort \textit{delay}. Nye værktøjer til at bekæmpe kriminalitet og skabe tryghed skal gerne skabe resultater hurtigst mulig for at bevise deres værd. PredPol reklamerer på deres hjemmeside med at mindske kriminalitet på få måneder~\cite[]{PredPol}. Dette gøres ved at ændre \textit{The Structure of Information Flows}, hvor information udledt fra Big Data transmitteres direkte til de personer, der skal opretteholde lov og orden.

Fjernes disse nye værktøjer, som PredPol eller CityPulse, fra systemet vil deres påvirkning på systemet øjeblikkeligt forsvinde. Kan ordensmagten ikke længere være til stede til at afværge de kriminelle handlinger, vil de opstå som før og utrygheden vil genopstå. Måske endda i endnu højere grad, da borgerne vil have mistet en del af deres ansvarsfølelse under ordensmagtens kontrol. 


\subsection*{[b]-listen}
I [b]-listen præsenterer vi en alternativ praksis, hvor vi gennem en \textit{reframing} af problemet~\cite[]{FrameInnovation} vil vise en anden tilgang til systemet: I stedet for at se tryghed som myndighedernes ansvar, skal tryghed tilhøre borgerne og fællesskabet. Inspireret af Dorrestijn \& Veerbeek~\cite[]{dorrestijn2013technology} vil dette blive et distribueret system, hvor ansvaret for trygheden ligger hos det enkelte individ og kun gennem fællesskab kan en samfundsrækkende tryghed skabes.

Dette kræver, at borgerne får de nødvendige værktøjer og information samt en fælles bevidsthed om et fælles ansvar. Denne indsigt opnås ved at flytte ejerskabet af den data, der genereres af moderne teknologi fra systemet til individet selv. Således flyttes også ansvaret for denne datas anvendelse til individet, og det er kun gennem aktiv deltagelse i et større fællesskab, at denne data bliver anvendelig\footnote{Dette kan i sagens natur ses som én stor tillidsøvelse borgere i mellem. Der er en overhængende risiko for at ekstreme bevægelser vil forsøge at få den endelige magt over samfundet. I bund og grund er dette et grundlæggende problem med socialistiske projekter og konsekvenser af dette har været gennemsyrende for vores globale samfund gennem det 20. århundrende.}.

Konsekvensen af dette system vil være en anvendelse af Big Data, hvor hvert enkelt individ vil blive hørt, fordi det i sagens natur er medejer af systemet. %I stedet for at behandle borgerne som rækker og kolonner i et system, bliver borgerne udgangs- og omdrejningspunktet for det fællesskab, der danner grundlaget for et trygt samfund. 
Bryder enkelte medlemmer af fællesskabet sammen, vil det resterende fællesskab løfte individet i flok, da det er afhængigt af det enkelte individ. Denne samfundsstruktur er manifesteret i [b]-listen nedenfor, hvor det ses sammenholdt med [a]-listen.

\begin{figure}[H]
    \centering
    
\begin{multicols}{2}
\begin{itemize}
    \sloppy
    \item[] \textbf{[a]}
    \item Tryghed oppefra
    \item Tryghed er tilliden til systemet
    \item Symptombehandling
    \item Skabe ro og orden
    \item Systemet beskytter individet
    \item Systemet afværger kriminalitet
    \item Systemet profilerer individer
    \item Data ejes af systemet
    \item Data bruges af systemet
    \item Kortsigtet~\textrightarrow~nu og her

    \item[] \textbf{[b]}    
    \item Tryghed nedefra
    \item Tryghed er tilliden mellem individer
    \item Skabe en fælles kultur
    \item Gro gensidig forståelse
    \item Fællesskabet oplyser individet
    \item Fællesskabet løser konflikter
    \item Individer hjælper hinanden
    \item Data ejes af individet
    \item Data bruges af fællesskabet
    \item Langsigtet~\textrightarrow~selvopretholdende
    
\end{itemize}
\end{multicols}
    \caption{[a]- \& [b]-listen}
    \label{fig:a_b_liste}
\end{figure}

\noindent Informationsflowet i denne nye virkelighed er altså radikalt anderledes end det tidligere. I stedet for at lade information flyde fra borgerne til de offentlige instanser, vil informationen flyde mellem borgenerne og skabe en bedre forståelse imellem dem. Effekten af overgangen til et sådant system vil have et langt \textit{delay} og ikke mindske kriminaliteten på få måneder. I stedet vil denne nye samfundsstruktur have en langsigtet og selvopretholdende effekt, hvor trygheden ikke vil være afhængig af enkelte faktorer, men indbygget i kulturen. På denne måde bliver målet at skabe et samfund og en kultur, hvor ingen vil begå kriminalitet.

\subsection*{En bedre forståelse}
%Vi har opstillet [a]- \& [b]-listen, der står som modsætninger til hinanden, hvor [b]-listen lige nu er en ren utopi, der ligger langt fra den virkelighed, vi lever i. Ikke rent teknologisk men kulturelt og menneskeligt. 
[a]- \& [b]-listen udspænder et spektrum fra det ene ekstrema til det andet. Som beskrevet tidligere har nye teknologier ført til, at vi har flyttet os mod en virkelighed, der minder mere om [a]-listen. Dette er sket og sker uden nogen offentlig debat om, hvorvidt det er den retning, vi ønskeligt vil bevæge os i. Teknologien kan sagtens bruges anderledes, men vi følger tilsyneladende nogle mønstre og tankesæt, der fører os i denne retning. Se illustrationen i figur[FIGUR].

\textit{Dataman} skal hjælpe brugeren til at bryde ud af disse ubevidste mønstre og tankesæt, til at se der er flere retninger vi kan bevæge os i og ultimativt til at tage aktivt stilling. Figur [FIGUR] illustrere dette i en repræsentation inspireret af Dunne \& Rabys diagram over mulige fremtider~\cite[]{Dunne:2013:SED:2613526}.

Det er styrkelse af det enkelte individ, hvor han selv kan vælge hvor meget data han genererer, hvilken data han vil generere, hvornår han vil generere data og hvad han vil bruge den til. Det er en lille lomme af kontrol i et samfund, hvor vi ellers sjældent har kontrol over vores data. Vi kan selv tilvælge overvågningen \cite[]{dorrestijn2013technology}. Det er en ekstrem overvågning af os selv, men vi er samtidig i total kontrol af denne overvågning. Vi behøver ikke at vente på omverdenen. Vi får mulighed for at tage aktive valg nu og her. Det er måske kun omkring vores egen lille lomme af data, men vi kan rent faktisk tage et valg og føre det ud i livet inde i denne lille lomme.

%HER HAR MAGNUS TILFØJET TING. LÆSES IGENNEM?
\section*{Socioæstetiske overvejelser}
Modsat andre prototyper (se afsnittet Designproces) er Dataman ikke afhængig af et fællesskab for at fungere, men optager altid data omkring brugeren. Dog kan brugeren drage nytte af fællesskabet. Når to brugere af Dataman mødes og giver hinanden hånden, udveksles en smule data mellem de to brugere og dermed bekræfter de hinandens data.

Håndtrykket har siden middelalderen været en gæstus af høj symbolsk værdi\footnote{\url{https://www.kristeligt-dagblad.dk/liv-sj\%C3\%A6l/vi-giver-h\%C3\%A5nd-til-ligem\%C3\%A6nd}}.

\begin{displayquote}[Inge Adriansen, PhD i Nordisk Folkemindevidenskab]
\textit{De sammenlagte hænder er et stærkt og letforståeligt symbol. Det er et tegn på gensidighed og et ligeværdigt forhold, da begge parter udfører den samme handling.}
\end{displayquote}

Også i dag er håndtrykket en vigtig del af det sociale liv. Håndtrykket er altafgørende for, hvilket førstehåndsindtryk, vi danner os om de mennesker, vi møder. 

Dataman udnytter denne symbolik og påfører den endnu et aspekt. Når du giver hånden til en anden person, eksponerer du en lille del af dig selv. Meget konkret får den anden person adgang til noget data omkring dig. Håndtrykket bliver derfor i endnu stærkere grad et symbol på tillid og gensidighed. Du ville vel ikke give en person, du ikke stoler på adgang til \textit{din} data? Og kan du egentlig stole på en person, der skjuler sin data fra dig? Dataman forsøger ikke at svare på disse spørgsmål, men giver i stedet brugeren mulighed for, at opdage og overveje disse problemstillinger selv.

Som beskrevet tidligere er indikerer Dataman sin optagen ved hjælp af en lysende diode. Grøn når der optages, rød når der ikke gør. På denne måde, kan omgivelserne nemt se, om brugerne af Dataman optager eller ej. Omgivelserne må så selv afgøre, hvordan de vil reagere på en sådan bruger. Ville du gå hen og snakke med en fremmed, du vidste optog jeres samtale? Ligeledes kan brugeren reagere på sine omgivelser. Hvis hun gerne vil kommunikere fortrolighed kan hun slukke optagelsen på sin Dataman. Modsat kan hun tænde den, hvis hun står i en utryg situation, hun gerne vil dokumentere.

Den data, Dataman genererer, kan tilgås gennem et web interface. Man har altså altid adgang til sin tidligere data. Denne data kan som beskrevet manipuleres, slettes, sælges eller doneres. Faktisk er der ingen grænser for, hvordan ens data kan bruges. Man ejer det jo selv. Hvad nu, hvis din kæreste spurgte, hvor du var henne i går? Hvad hvis hun ikke stolede på, hvad du sagde og forlangte at se data, der bekræftede det? Hvad hvis du have slettet den data? Ellers solgt den?

Dataman følger med, hvorend brugeren går. Derfor indgår den også i flere forskellige sociokulturelle kontekster og vil have forskellige betydninger i forskellige kontekster. Dette stemmer ens med \cite[]{Petersen:2004:AIP:1013115.1013153}, der netop argumenterer for, at den æstetiske oplevelse ved et designartefakt ikke kun kan ses i isolation, men skal forstås i en brugskontekst. Vi vil altså argumentere for, at Dataman i høj grad er æstetisk i en sociokulturel forståelse af begrebet.

%Der opstår nogle sociale overvejelser. ... æstetisk interaktion...

%Håndtrykket??

\section*{Forskellige typer tryghed}
Til at arbejde med forskellige typer af tryghed har vi opstillet et diagram, der viser, hvem der tager initiativet til at skabe tryghed, og hvordan denne tryghed skabes. Trygheden kan enten skabes gennem kontrol eller gennem tillid og initiativet kommer enten fra individet eller fra fællesskabet. Fællesskabet kan i nogle tilfælde være repræsenteret af andre. I [a]-listen er det fx systemet, der varetager fællesskabets interesser. Diagrammet kan ses i figur [FIGUR], hvor [a]- \& [b]-listen også er plottet ind. 
    


\begin{comment}
 

%Det sjove videokort

%Det handler om overvågning
%Smart cities generelt
%NOGET OM POL INTEL!
%opdater a og b listerne til det nyeste
%Social inoovation (Nicholls and mudic tekst)
%De to akser symboliser forskellige former for tryghed
\end{comment}