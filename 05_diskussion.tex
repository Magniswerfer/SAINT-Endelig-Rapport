\section{Diskussion}

\subsection{konceptet}
Sikkerheden i Dataman skulle adresseres stærkt. Der skulle være en stærk End-to-end kryptering mellem enheden og brugerfladen. Teknologien er i dag stærk og udbredt nok til at det ville kunne lykkedes, så dataen var stærkt beskyttet. Man kunne også have valgt at lave en ren trådet forbindelse mellem enheden og ens brugerflade, som måske ville have givet et stærkere æstetisk indtryk, og fornemmelse af at man virkelig har styr på sine data. På den anden side, skulle produktet placeres og forestilles i vores moderne verden, ville det være stærkest med sin trådløse forbindelse mellem alle vores andre trådløse enheder. en trådet forbindelse ville heller ikke forhindre rent fysisk tyveri af sine data, hvor omvendt kan en trådløs forbindelse gøre dette umuligt. 

Håndtrykket for at udveksle data giver også tryghed og en sikkerhed, i og med man fysisk skal have enhederne meget tætte på hinanden for at udveksle denne data. 

I og med 


\begin{comment}
TING AT DISKUTER
    Bruge den teknologi vi har i dag (trådløs)
    
    Hvordan vil dataman ændre vore hverdagoplevelser. Hvordan kandet bruges på nye måder
    HVor er dataman ikke æstetisk (i sin intraktion)
\end{comment}

