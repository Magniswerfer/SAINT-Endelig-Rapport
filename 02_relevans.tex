\section{Relevans}
Vi mener tryghed er en essentiel menneskelig følelse alle har ret til. Der er en direkte sammenhæng mellem tryghed og mentalt velvære. Når vi føler os utrygge er vi i beredskab, vi kan ikke slappe af, utrygheden tager vores fokus og kan potentielt overskygge alt andet. Kigger vi på Maslows behovspyramide~\cite[]{maslow}, ses det at tryghedsbehovet ligger tæt på nederst kun med fysiske behov under sig. Føler vi os ikke trygge falder behovet for det sociale, for vores ego og selvrealisering i baggrunden og vores energi bruges på at skabe tryghed, sikkerhed og stabilitet for os selv. For at et moderne samfund skal fungere og vi som mennesker, samfund og civilisation kan bevæge os fremad teknologisk og socialt, må vi føle os trygge. Trods dette må vi stadig overveje, hvordan vi skaber denne tryghed, hvilken form for tryghed vi er ved at skabe, hvem vi skaber tryghed for og på hvilken og hvis bekostning trygheden bliver skabt. 

%Social inno litteratur?

Der er i de seneste år flere eksempler på teknologier, der bruger Big Data til at overvåge det offentlige rum og bekæmpe kriminalitet og uorden for derigennem at skabe tryghed for samfundets borgere. Dette skaber nye problematikker og potentielle virkeligheder~\cite[]{Dunne:2013:SED:2613526}, vi som borgere i samfundet er nødt til at forholde os til. Nye teknologier gør det muligt at indsamle nye former for data. Fx indsamles der information fra sociale medier, der kigges på tidligere kriminelle begivenheder i geografiske områder og der optages lyd i bybilledet. Samtidig sættes flere og flere overvågningskameraer op. Dette skaber en enorm mængde data, som politiet eller andre kan tilgå. De kan dog umuligt analysere det manuelt. Derfor bliver data i højere grad analyseret af algoritmer, der hurtigt kan analysere og finde mønstre i de store datamængder. Dette har ført til en række produkter og løsninger, der er kommet på markedet de sidste par år. Fælles for disse løsninger er, at de symptombehandler uro og kriminelle tendenser. Big Data bliver brugt prædiktivt præventivt i opstående og eskalerende situationer og går generelt ikke dybere end dette. 

Eksempler på dette er CityPulse i Eindhoven~\cite[]{CityPulse} og SCAN NET i England~\cite[]{SCANNET}. CityPulse bruger realtidstracking af nattelivet med kameraer og mikrofoner samt realtidsanalyse af sociale medier som Facebook og Twitter til identificere optakt til uroligheder og stoppe dem i optakten. Fx ved at lyse området op. SCAN NET registrerer alle tilmeldte natklubbers besøgende med fingeraftryk og billede i en samlet database. Hvis en person får karantæne eller bliver smidt ud fra én natklub kan denne information tilgås af andre natklubber tilmeldt SCAN NET og dermed kan man udelukke individet fra samtlige natklubber baseret på en forseelse ét sted.

Et af de helt store buzzwords lige nu inden for bekæmpelse af kriminalitet er \textit{Predictive Policing}, hvor eksisterende data fra kriminalregistre, demografi, vejret og andet data bruges til at forudse, hvorhenne, i hvilket tidsrum og hvilken form for kriminalitet, der er mest sandsynlig. Et eksempel på sådan et system er PredPol~\textit{PredPol}, der anvendes i flere amerikanske storbyer til fortælle politiet, hvor de skal yde deres indsats og være synlige. I Danmark vil rigspolitiet begynde at tage Pol-Intel i brug i løbet af 2017. Pol-Intel er et Predictive Policing-system til at analysere kriminalregistre og sociale medier. Det er udviklet af Palantir, der også udvikler overvågningssytemer til bla. NSA, CIA og det amerikanske militær~\cite[]{PolIntel}. Politichef for rigspolitiet Svend Larsen argumenterer for, at politiet allerede i forvejen gør disse ting og at Pol-Intel blot effektiviserer det~\cite[]{Aflyttet}, men er det en valid argumentation, når effektiviseringen sker med en faktor 1000?


\begin{comment}

\end{comment}
